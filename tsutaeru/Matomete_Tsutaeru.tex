\documentclass[12pt,a4paper]{jsarticle}
%
\usepackage{amsmath,amsthm,amssymb}
\usepackage{atbegshi}
\AtBeginShipoutFirst{\special{pdf:tounicode 90ms-RKSJ-UCS2}}
\usepackage[setpagesize=false,
bookmarks=true,	%しおりを作る
bookmarksnumbered=true,	%しおりに節番号などを付ける
bookmarksopen=true,
pdftitle={},%
pdfauthor={佐々木裕},%
pdfsubject={サブタイトル},%
pdfkeywords={キーワード},
%linkcolor=blue,anchorcolor=blue,urlcolor=red,
colorlinks=true,linkcolor=blue,filecolor=blue,urlcolor=red,
dvipdfmx]{hyperref}
%
\usepackage{multirow}
\usepackage{bm}


\usepackage[dvipdfmx]{graphicx}%
%\def\pgfsysdriver{pgfsys-dvipdfmx.def}%(graphicx パッケージを使用しない場合はこの行を有効に)
\usepackage{tikz}%(これで、pgf と pgffor も読み込まれます。)
\usetikzlibrary{positioning}


%\def\pdfliteral#1{\special{pdf:content #1}}
%\usepackage[dvipdfmx]{color}
%
\usepackage{wrapfig}
\usepackage{ascmac}
%\usepackage{plext} 
%\usepackage{epsdice}
%\usepackage[dvipdfm]{pict2e}
%
\def\diff{\mathrm d}
\def\dd#1#2{\dfrac{\diff #1}{\diff #2}}
\def\pp#1#2{\dfrac{\partial #1}{\partial #2}}
\def\dd2#1#2{\dfrac{\diff^2 #1}{\diff #2^2}}
\def\pp2#1#2{\dfrac{\partial^2 #1}{\partial #2^2}}
%
\allowdisplaybreaks[3]
%
\makeatletter
%\def\section{\@startsection {section}{1}{\z@}{-3.5ex plus -1ex minus % -.2ex}{2.3ex plus .2ex}{\Large\bf}}
\def\section{\@startsection 
{section}
{1}
{\z@}
{3.5ex plus -1ex minus -.2ex}
{1.5ex plus .2ex}
{\large\bf}
}
\makeatother
%
\makeatletter
\def\subsection{\@startsection 
{subsection}
{1}
{\z@}
{3.5ex plus -1ex minus -.2ex}
{1.ex plus .2ex}
{\large\bf}
}
\makeatother
%
\setlength{\textwidth}{\fullwidth}
\setlength{\textheight}{40\baselineskip}
\addtolength{\textheight}{\topskip}
\setlength{\voffset}{-0.2in}
\setlength{\topmargin}{0pt}
\setlength{\headheight}{0pt}
\setlength{\headsep}{0pt}

\title{「考えをまとめて、人に伝える」}
\author{佐々木裕}
\date{\today}

\begin{document}
\maketitle

\section{はじめに}


\subsection{日々の活動において}

皆さんの日々の活動において、他の人に「自分の考え」を「伝える」機会は数多くあります。

たとえば、朝のミーティングのような日々の連絡会や、月報会のような実験結果の報告会のような場面です。
その際、その場の状況に見合った時間で、自分が伝達したい内容を効率よく伝える必要があります。

\subsubsection{意思の疎通は良好ですか?}

ご自分の過去を振り返ってみて、「私は上手に意思の疎通ができている」と自信のある方がどれほどいるでしょうか?

普段の雑談で適当な受け答えをしているときうまく話が通じているような相手であっても、正確な話をしようとすると途端にうまくいかなくなることはありませんか?

そして、ちゃんと説明しようとして詳しく話そうとすればするほど、逆に、訳が分からなくなってしまって意思の疎通がうまくできないという困った状況になった方も少なくはないでしょう。

\subsubsection{自分の考えのありかは?}

このような状態に陥った時に、「考えがまとまらない」ということを実感されるのではないでしょうか。

胃や心臓の活動のような不随意筋の運動ならいざ知らず、一般には、行動や行為は、行為者自身の考えに基づいてなされているものでしょう。
確かに、「つい無意識で」というような行動もあるかもしれません
でも、意識しているかどうかの違いはあれ、本来、それぞれの行動にはそれと連動した自身の考えがあるはずです。

しかしながら、改めて自分の考えを整理しようとしてみると、自分の「思いのありか」を突き止めることも難しいし、さらにそれを「まとめる」ということはとても困難だということに気づくでしょう。

\subsubsection{伝えるべきポイントは?}
さらに、上記の伝達の場面において、「話のポイントが不明確」で自分が何を伝えるべきなのかがよく判らなくなることはないでしょうか?

上記のように、ご自分の考えがまとまっていなくてその流れが不明確な場合は、当然、伝えるべきポイントは明確ではないでしょう。しかし、人が書いた文章を説明や代読しようとする場合にでも、著者の伝えたいポイント(要旨)を明確に提示できる自信を持たれている方もそれほど多くないのではないでしょうか。

つまり、伝えるべきポイントを明確にするためには、伝えたい内容の全体像をきれいに整理した後に、その中から重要な部分を抜き出すという「要約」という行為が必要なのです。
自分の考えをまとめるだけでも大変なのに、その上に、全体像の把握と重要なポイントの抽出を行うということは、非常に困難であると言わざるを得ないでしょう。

\subsection{「考えを上手に伝える」ことの価値}

「考えをまとめる」ことと、「人に伝える」という行為を組み合わせると、仕事上でどのような良いことがあるのでしょうか。

自身の考えをまとめることができれば、その時点での自身の行為を総括でき反省すべき点が明確になり、自身のおかれている状況を把握するためにも役立つものと期待できます。
そして、その考えを「人に伝える」ことが上手に行えれば、自分の周りの他人と意思の疎通(コミュニケーション)が良好に図れるようになるでしょう。

\subsubsection{具体的な価値}

仕事を円滑に進めるために、上司や同僚との間での「ほうれんそう:報(告)、連(絡)、相(談)」が大事であると言われています。
この際に、上記の二項を意識して自身が置かれている「状況を正しく把握して、適切に伝達」することで、適切な指示やアドバイスをもらうことができるものと期待できます。
また、月報会等の議論の場においても自身の実験した内容や結果について「的確な整理を行ってから、順序良く発表」することができれば、当を得た議論を行うことができて自身の考え方の妥当性を確認できます。

\subsubsection{トレーニングの勧め}

ここに示したように、「考えをまとめて、上手に伝える」ことの効果は非常に大きなものでありますから、その行為を的確に行うために頑張ってトレーニングを積み重ねるだけの意義と価値はあると思います。
以下に、そのトレーニング方法についての具体的な提案を示します。

\section{「考えをまとめる」}

前章において、皆さんのコミュニケーション能力に関する勝手な憶測を書かせていただきました。「思い当たることはまったくなくて、日々のコミュニケーションには問題なし。」と胸を張れる方なら、この先を読む必要はないでしょう。

でも、胸に手を当てて考えてみて何か少しでも思い当たる点があった人は、これ以降の文章が少しは役に立つかもしれません。

\newpage


\subsection{困った状態}

「考えがまとまらない」や「話のポイントが不明確」というような状態は、なぜ生じてしまい、どうすれば改善することができるのでしょうか。
前章に示したように、考えをまとめるという行為はなかなかに難しいことです。伝えたい概念が、まとめられて(整理できて)いない不明確なものであった場合、それを伝えようとすることは非常に困難です。
未整理な内容を無理に言葉にしようとしても、混乱して頭の整理がつかないのは当たり前です。多様な事柄の整理は簡単にはできません。その結果として、望んだ行為と真逆な状態である「逆上して支離滅裂なことを言い出す」とか、「黙り込む」というような状態に至るのであれば、本末転倒です。

\subsection{一般的なやり方}

\subsubsection{紙での作業}

皆さんは、考えをまとめるためにどのような方法をとられているでしょうか?
ただ座って、頭の中で考えを巡らせる人もいるでしょうが、たいていの場合、すぐに煮詰まってしまって整理がつかなくなるのではないでしょうか。
やっぱり、メモ用紙でも使って、何か書きながら整理する方がよさそうですね。でも、このような行為を紙の上で行っていると、結構疲れてしまいませんか。だって、紙は書いた文書の入れ替えなんてできませんから、上手に書かないとすぐにグチャグチャになってしまいます。そして、朱で直しても最終的には「清書」という果てしない作業が待っています。

\subsubsection{コンピューター作業}

そういう時にこそ、コンピューターを有効活用すべきです。
この作業をWORDで行っている人も多いでしょうし、ExcelやPowerPointでやる人もいるでしょう。
確かに文字を消したり入れ替えたりはできますから、ずいぶん都合がよさそうですが。

ただ単に見栄えの良いものを作ったり、データをペタペタ張り込んだりするだけなら、これで十分でしょう。
しかしながら、一般には、これらのソフトでは、全体の見通しがよくありません。
したがって、部分と全体を行き来することが困難であり、全体像を見失いがちなのです。
全体像を把握する事こそが「考えをまとめるためにもっとも重要」なのだと、私は考えています。

\subsection{あるべき姿に向けて}

では、どのような状態になれば、自身の考えを上手にまとめることができるのでしょうか。
キーワードを二つ挙げるとすれば、以下の二つでしょう。

\vspace{5mm}
{\LARGE 「理路整然」と「一目瞭然」}

\subsubsection{説明(理解)のためには一本道の流れ}

他人の書いた文章を読むことを想像してみましょう。
色々な思考の流れが交じり合ってしまって、主たる流れが明確でない文章は、読んでいても何が書いてあるのかを理解することが大変です。
書かれた文章を読むときには行ったり来たりしながら読むこともできますが、流れが交じり合っていると、どの道を通っているのかすら判らなくなって迷子になってしまいます。
ですから、他人に自分の思考の流れを理解してもらうためには、一本道で単純な流れを提示することが重要となります。
つまり、文章を作成する側には、聞き手の理解を助けるために「容易に追跡できる明確な一つの流れを生み出す」ことが要求されます。

\subsubsection{文章の構造化}

一本道の思考の流れさえ明確になれば、人に説明しやすい状態ができあがるわけではありません。
一般に、まとまった文章を理解しようとするとき、人間はその文章の内容をいくつかの「意味のまとまり」に区切るとともに、それら同士の関係を手がかりにしながら、文章全体の意味を把握しようとします。
つまり、長い文章を意味のつながりに基づいて切り刻んでいき要約して、見出しをつけて粗筋を理解しようとするわけです。
逆に、このような要約が一目で理解できるように、意味のまとまりに基づいた適切な見出しがついて、それぞれの部分の繋がり方が明確になった文章を構造化された文章と呼びます。
構造化というと難しそうに聞こえますが、要は文章同士の関係を明らかにしながら、論理の流れが判りやすいように全体の構成をつくりあげていくことです。
こうすることで、理(ことわり)の路を整然としたものとして示すことができると思います。

\subsubsection{一目で見渡せる状態}

この時、「見出し」は、読み手に文章の「意味のまとまり」を適切に把握させるための指標として機能します。したがって、見出しとなるような項目が順序良く並んだ状態を作り出したいわけです。
さて、こう書いてみると、本や論文の最初についているもので、何か思い当たりませんでしょうか。
そうです、目次です。
結局、自分の考えをまとめるための文章を書く際に、同時に目次を作成できれば、自分の考えの全体像を一目ではっきりと見渡すことができるわけです。

\subsection{処方箋}

前述の「起こりがちな(困った)状態」を「あるべき姿」へと改善するためには、どのようにすればいいのでしょうか。
現時点での私のオススメは「アウトラインの有効活用」であり、トレーニング方法としての「論説文の要約」ということになります。

\subsubsection{コンピューターの有効利用}

ちょっとでも長い文章を書いたことのある人ならお分かりと思いますが、最初に章立てを行っても、書いているうちにその章立てを変化させた方がいいと思えてくることがよくあります。
逆に溢れ出るアイディアを章立てで押さえ込むこともあります。
記述していく順序にしても、頭から書くというわけでもありません。
結論となるべき終わりの章から書き始めて、一番最後に序章を書くというのも常套手段ですし、むしろその方がよいと言われてもいます。
また、推敲段階で文章の順序を入れ替えることもよくあります。
要するに書くときにも、読むときと同じように未完の流れの中を行ったり来たりしながら、部分と全体を行ったり来たりする必要があるわけです。

\subsubsection{「アウトラインの有効活用」}

アウトライン(Outline)という言葉の元々の意味は、図形の輪郭(外側の線)であったと思いますが、そこから転じて、お話の概要とか文章構成というような意味になっています。
私からの提案は、このアウトラインを作成することにより、「自身の考えを見通しの良い状態の下で整理しよう」ということです。
この作成のためには、専用のソフトウェアも存在しており、アウトラインプロセッサと呼ばれています。
確かに専用ソフトの方がいろいろと使い勝手はいいのですが、WORDに付属のアウトラインモードでも、最低限必要とすることはできます。

\subsection{具体的なトレーニング}

「容易に整理できるような事柄を対象」として選択することが、一番困難なハードルかも知れません。逆に言えば、対象をこのように整理して箇条書きができれば、ほぼ、考えがまとまったということができます。

\subsubsection{アウトライン形式の利用:}

全体の構成を先に決めて、「トップダウン形式で論理の流れを構築する」という流れで、アウトラインモードに慣れましょう。
この時、「表示」メニューの下にある「ナビゲーションウィンドウ」にチェックマークを入れると、左側に目次と類似な表示が出てきます。

アウトライン形式で考えを整理する方法について、以下にまとめました。

\begin{enumerate}
\item
最初にラフな設計図を決める。
	\begin{itemize}
	\item
	暫定的なタイトルを決める。
	\item
	ラフなストーリーを決める。
	\end{itemize}
\item
第一レベルの項目を「適当に」決める。
	\begin{itemize}
	\item
	論説文形式の場合、「イントロ」、「本文」、「結論」等。
	\end{itemize}
\item
第二レベルの内容を記述する。
	\begin{itemize}
	\item
	出口の項目(結論)、あるいは、入口(イントロ)から始めると分かりやすい。
	\item
	上記と逆の項目について記述する。
	\item
	「本論」の部分は、後回しにした方がよい。
	\end{itemize}
\item
第一レベルの項目を書き換えていく。
	\begin{itemize}
	\item
	下位レベルの記述内容を勘案し、上位レベルの表現を見直す。\\
	第二レベルの記述と、全体の流れを見ながら。
	\item
	上記のサイクルを繰り返す。
	\end{itemize}
\item
細かい記述の付加(より細い小骨をつける)。
	\begin{itemize}
	\item
	それぞれの項目に対して、下位レベルの記述をおこなう。
	\end{itemize}
\item
ストーリーの背骨を明確にする。
	\begin{itemize}
	\item
	第一レベルの記述を短文で書いてみる。
	\item
	これを続けて読むことが、もっとも短い概要となる。
	\end{itemize}
\end{enumerate}

\subsubsection{トレーニングとしての「論説文の要約」}

文章の流れを明確にとらえるための訓練として、「論説のような文章」を題材として選択し、その文章を「要約」するという方法を提案します。なかなか捉えづらい自身の思考ではなく、他人の文章から思考の流れを読み取ろうということです。

つまり、著者の主張を明確に表せるような「要約」を作成し、それを基にして「著者の考えを整理」します。そして、前述のようなトップダウン形式のアウトライン方式により、「短文の組み合わせ」で、「単純で判りやすいストーリーになるように書く」というトレーニングです。
このようなトレーニングにより、「整理しやすい対象からシンプルなストーリーを作る」ことを繰り返すことで、段階的に、複雑な事象の整理技術の向上が望めるのではないかと期待しています。
要約を作成する際に気を付ける点を以下に列記しました。

\begin{itemize}
\item
文章の全体像を理解する。
	\begin{itemize}
	\item
	何度も文章を音読する。
	\end{itemize}
\item
著者の考えの流れを明確にする。
	\begin{itemize}
	\item
	文章の意味合いに応じて、段落程度の大きさのいくつかのブロックに分ける。\\
	(たとえば、起承転結の様に。)
		\begin{itemize}
		\item
		それぞれのブロックの中で、再度、数文章程度の大きさで分割する。
		\item
		著者の考えを理解するために必要と思われるキーワードに下線を引く。
		\item
		その数行をできるだけ短い「名詞句」で表現してみる。
		\item
		流れを表現するのに必要な名詞句を選択する。
		\item
		選んだ名詞句を並べて、そのブロックを短い文章で表現する。
		\end{itemize}
	\item
	ブロックの流れを明確にする。
	\end{itemize}
\item
できるだけ短い文章でそれを表現する。
	\begin{itemize}
	\item
	上記の流れを意識する。
	\end{itemize}
\end{itemize}


\subsubsection{文章の手直し}

考えをまとめるために書いている文章でも、校正や推敲を確実に実施する必要があります。
\begin{itemize}
\item
思考の流れのチェック。
	\begin{itemize}
	\item
	アウトラインの高いレベルでの流れのスムーズさをチェック。
	\item
	短い文章で概要が話せるようにする。
	\end{itemize}
\item
文章を音読する。
	\begin{itemize}
	\item
	思い込みを排除するために、文字を正しく読む。
	\item
	自身の発した言葉を耳経由でフィードバックし、頭の中で反芻。
	\item
	誤字、脱字がないかをチェック。
	\end{itemize}
\end{itemize}

\section{「考えを上手に伝える」}

\subsection{伝えるためには}

\subsubsection{伝達行為は双方向}

「人に伝える」という行為は、一人で行うことではなく相手が存在してはじめて成り立つ行為です。
実際の報告の場を考えてみましょう。
会社では上司や同僚であったり、また、部下に対する指示の場でも生じますし、大学であれば先生への報告や学生同士での議論や後輩への指導の場であるかもしれません。

したがって、伝達する場合には、自身の立場だけではなく、相手の都合や状態も考慮する必要があります。

\subsubsection{立ち位置を変えてみる}

自身の側に立った場合、「伝える」という行為は、自身の頭の中にある「伝えたい内容」を、「適切な言葉で示す(話す)」という風に分解できます。
また、受け取る立場に立った場合のことを推し量る必要もあるでしょう。
ここで、自分が誰かから報告される場合のことを想像してみましょう。
その場合、「どのような内容の話」であるかとか、「その内容について(自分は)どのぐらいの知見があるか」を想定して、話を聞くのではないでしょうか。
つまり、話者としては、受け取ってくれる相手が「どのようなことを聞きたいのか」や「どんな知識を有しているのか」というようなことを推測しなければいけません。
さらには、「相手が理解しやすいであろう順序」もよく考慮する必要もあるのです。

\subsubsection{伝えるという行為}

結局、「人に伝える」ということは、自身の頭の中にある「伝えたい内容」を、「他人が理解しやすい順序に並べて」、「判りやすい適切な言葉で示す(話す)」ということになります。
こうすることで、自身の考えを他人に的確に伝達できるようになるわけです。

\subsection{処方箋}

「伝える力」の向上については、正直に言って、未だ模索段階です。
しかしながら、結局は考えをまとめるという行為と不可分であろうと思いますので、アウトラインを利用して、努力することが一番良さそうです。
現時点で、考えているような内容について、以下に列記しました。

\subsubsection{考え方の確認}

\begin{itemize}
\item
話す対象を具体的に想定する。
	\begin{itemize}
	\item
	イメージトレーニングを行う。
	\item
	話の流れを客観的にとらえる。
	\item
	聞き手の立場で考えてみる。
	\end{itemize}
\item
あらすじを明確にする。
	\begin{itemize}
	\item
	考えのアウトラインを明確にする。
	\item
	アウトラインの上位レベルを提示する。
	\item
	適切な概要を最初に伝える。
	\end{itemize}
\item
心づもりをしてもらう。
	\begin{itemize}
	\item
	キーワードとなる概念を明確に。
	\item
	一般的な共通概念を利用する。
	\item
	判りづらい表現。
		\begin{itemize}
		\item
		できるだけ避ける。
		\item
		上手な言いかえを考える。
		\end{itemize}
	\item
	キーワードの数を多くしすぎない。
	\end{itemize}
\end{itemize}

\subsubsection{アウトラインを利用したパワーポイントの作成}

前述の処方箋に列記した手法を実践して、皆さんのコミュニケーション能力が高くなれば、各種の有効な伝達方法を実施することができるでしょう。
しかしながら、現状の実力(未だ低いかもしれない?)の下でも、パワーポイントを用いたプレゼンテーションを行う必要も生じるでしょう。
そのような状況では、前述のアウトラインを有効に利用することが大事だと思います。
以下に示したような処理により、同一のアウトラインから、「ワードの文章」と「パワーポイントの箇条書き」を作成することができるようなります。

ワードのアウトラインからパワーポイントへ
\begin{itemize}
\item
ワードにおいて
	\begin{itemize}
	\item
	アウトラインを作成し、セーブする。
	\end{itemize}
\item
パワーポイントにおいて
	\begin{itemize}
	\item
	新規文章の作成
	\item
	「開く」コマンドで、ファイルタイプとしてアウトラインを選択
	\item
	上記のアウトラインを開く。
	\end{itemize}
\end{itemize}

\section{上級編}

三章までの記述においては、伝えたい内容が比較的簡単な場合を題材にして、トップダウン形式で纏める方法について説明してきました。
ここでは、もう少し難しい文章を書く際に必要になる事項を挙げてみました。

\subsection{トップダウンが難しそうなとき}

\subsubsection{曖昧とした概念を伝える}

私自身の整理を兼ねて、私が自分の考え(たとえば、自分の判断にも色々な幅があってその論理の流れをどのように説明すれば適切なのかが明確ではないドキュメント)を発表するときに行う準備のやり方を抽象的に書いてみます。

まず、自分の頭で考えるということは、「考えたい対象」に関して、「心に浮かんでくるさまざまな由無し事を仮想的な空間の中でフワフワと浮かばせてみて似たようなものを近くに寄せ集めていく」というようなイメージになります。
そして、「考えをまとめる」という行為は、「アイディアという捉えどころが明確でないフワッと(曖昧模糊と)した概念」を「頭の中にある引き出しの中にきちんと詰め込んで整理整頓した状態に落ち着かせる」ようなものと考えることができます。
そのうえで、「ポイントを明確に」するということは、「整理された引き出しのそれぞれに適切な見出しをつけていく」というような行為と捉えることができます。

以下に、それぞれの段階ごとに、どのようなことに気を付けるべきかという観点から考察してみましょう。

\subsubsection{アイディア出しのプロセス}

人間の思考というものは、自分の頭の中では、いろいろな流れ方をすることが可能です。
ボーっと考え事をしている状態を考えてみましょう。
こんなときには、断片的な「思い」が浮かんでは消え、お互いにぶつかり合い、そしてどちらにともなく流れていき頭の中に渦巻いてしまうことがよくあります。

このような「行ったり来たりの自由な思索」を行っていると、普段の固定観念の下では決して出てこないような、自由な発想が可能になるのかもしれません。
頭の中に浮かんだ多種多様な曖昧なイメージを羅列してみることが、素材の洗い出しになるでしょう。

そうして、それらの思い付いた単語を似た物同士を寄せてみたり、対立するものを向い合せたりして、並べ替えたりしながら結び付けていくわけです。
そのような過程において意識して考え方の視点を変えてみると、それまで思いつきもしなかったことが、突然、明確な形を持った思念として浮かび上がることもあります。

このような状態が、「アイディア出し」には必要と考えられますから、一歩間違えれば、発散しがちな行為であると言えます。
したがって、最初の段階である「アイディア出し」は、後に続く、「まとめる行為」とはちょっとだけ色合いの異なる行為と捉えることができます。

\subsubsection{ボトムアップ形式}

アイディア出しからまとめていくときには、トップダウン方式ではなく、ボトムアップを行う必要もあるでしょう。
この時に気を付ける点について、以下に列記しました。
ここに記述した内容は、後述の構造化文章のためのパラグラフ作成とリンクしています。

\begin{itemize}
\item
キーワード出し。
	\begin{itemize}
	\item
	頭に浮かんだキーワードを列記。
	\item
	まずは、自由にたくさん出す。
	\item
	類似性や対句を意識して並べ替え。
	\end{itemize}
\item
トピック作り。
	\begin{itemize}
	\item
	キーワードの組み合わせで名詞句(トピック)を。
	\end{itemize}
\item
トピックセンテンス
	\begin{itemize}
	\item
	名詞句から一、二行におさまる短文。
	\end{itemize}
\item
パラグラフ
	\begin{itemize}
	\item
	短文の組み合わせによる記述。
	\item
	短文を組み合わせて、数行の一段落を書く。
	\end{itemize}
\item
構造化文章へ
	\begin{itemize}
	\item
	流れにあった接続詞を選択することに留意。
	\end{itemize}
\end{itemize}

ボトムアップ形式にならざるを得ない場合でも、簡潔な文章で部分を表現することに気を付けて、全体像を明確にできるようにしていけばいいと思います。

\subsubsection{まとめるということ}

結局、まとめるということは、「理解しやすい一本道の論理に基づく構造化された文章の作成」というあたりになろうかと思います。
もっと具体的に言ってみましょう。
まとめるという行為は、曖昧な概念を「単純な言葉で言い換え」て、それらの前後関係を意識しながら「論理の流れを一本道」にして、「理解しやすい見出し」をつけた状態に落とし込むということになろうかと思います。

そして、ポイントの明確化とは、上記のまとめで整理された見出しを「客観的に見直す」ことで全体的な論理の流れを「読み返しやすい」状態に捉えなおして、自分の考えの中でもっとも伝えたい「ポイントを抽出」することと捉えることができます。

\subsection{構造化された文章の作成}

ここでは、構造化された文章を作成するために有効な方法であるパラグラフ(段落)を構成単位とした考え方を説明します。

\subsubsection{パラグラフを作成する}

パラグラフとは、内容的に関連のある複数の文の集まりで、それによって一つの考え(トピック)を表現するものです。
一つのパラグラフが一つの意味のまとまりを表しており、複数のパラグラフを論理的につないだものが文章全体となります。適切なパラグラフを作成することが、構造化された文章を書くための第一歩です。

\subsubsection{パラグラフ作成時の注意点}

パラグラフを取り扱う時に注意すべき点について、以下に列記しました。

\begin{itemize}
\item
1つのパラグラフには1つのトピック
	\begin{itemize}
	\item
	二つ以上のトピックを1つのパラグラフに書くと判りにくい。
	\item
	同じトピックを改行して別のパラグラフに分割してはいけない。
	\end{itemize}	
\item
トピックセンテンス
	\begin{itemize}
	\item
	トピックについての説明は,一つの文(トピックセンテンス)に。
	\item
	トピックセンテンスは,主張したい内容を端的に表現するべき。
	\end{itemize}
\item
トピックセンテンスとその展開文
	\begin{itemize}
	\item
	パラグラフはトピックセンテンスとそれを支えるいくつかの文。
	\item
	トピックセンテンス以外の文は,
		\begin{itemize}
		\item
		トピックセンテンスの主張をより具体的に説明。
		\item
		他のパラグラフとの関係を明らかにするための文。
		\end{itemize}
	\end{itemize}
\item
トピックセンテンスの位置\\
トピックセンテンスは、パラグラフの冒頭にあると読みやすい。
	\begin{itemize}
	\item
	日本語では、もっとも伝えたい内容がパラグラフの最後に来がち。
	\item
	主張を明確に伝えるためには、第一文にトピックセンテンスを。
	\end{itemize}
\end{itemize}

\subsubsection{見出しをつける}

小さな意味のまとまりであるパラグラフを、その内容に応じて論理的に結合することによって大きな意味のまとまりが作られたら、「見出し」をつけます。

見出しは、読み手が文章を読み進めていくための指針(どのような内容についての文章なのか、事前に見出しを読むことによって予測可能)となります。
そのため、見出しは文章の意味内容を適切に示す必要があります。

\subsubsection{相互関係の明確化}

それぞれのパラグラフの相互関係を明確にしておくことで、構造化が容易になります。また、構造を一目で理解できるように、視覚表現によって可視化することも有用でしょう。
一般によくつかわれる相互関係について、以下に列記しました。

\begin{description}
\item[並列]
同じ重みを持った複数の情報が、並列に存在する場合です。提示順序に制約はなく、それぞれの情報は対等の重みを持っています。それを示すためには、番号を付けずに列挙するのが適切です。文書中にずらずらと列挙して並べるのではなく、「・」を使って項目をひとつひとつ提示するべきです。
\item[順列]
個々の情報の提供順序が規定されている場合であり、それらの提示順序を誤ると、文書として伝達の目的を達成できなくなってしまいます。ですから、その優先順位を明確にするため番号をつけて、順列であることを読み手に意識させる表現手法を取るべきでしょう。
\item[分岐]
条件によって、必要となる情報が異なってくる場合です。 中心となるのは「どのような条件によって分岐が起こるのか」というような情報ですから、条件を中心に表現して、「~の場合」というように具体的な分岐条件がわかるような見出しを付けるべきです。
\item[因果]
それぞれの情報が、原因→結果という因果律で結びついている場合です。この因果関係を明確に表現するには、矢印を使って視覚化する、といった工夫も効果的です。
\end{description}


\end{document}